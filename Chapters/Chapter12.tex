\chapter{Further work}
\label{ch:further}
\section{UBX limitations}
The UBX framework allows to write blocks with well defined separation of concerns, but does not actually \texttt{require} it: a coder could, for example, make a c-block store data in its private memory space, or make a i-block process the data with a computational behavior.\\
Moreover another limitation of UBX is that it still does not support blocks composition, while it should be possible to define sub-nodes to be used as regular blocks.\\
A new library named AB5C is in a very early developing stage and should replace UBX and solve those problematics.

\section{Deploy}
Deploy of the UBX node in a real-time environment as stand-alone application is trivial, but it could be necessary to integrate the node with other software components.\\
Orocos if yet another software framework for the real-time execution and communication of (mainly) robotics applications and can be used to run UBX nodes alongside other processes, moreover it embeds a Lua environment in the \texttt{ocl} package that can be used to parse the UBX configuration files and scrits; however ocl is linked to the native Lua libraries that does not support the FFI library necessary to call C functions from Lua.\\
To demonstrate the feasibility of such set-up, Orocos ocl sorce code has been forked and modified in order link against the LuaJIT libraries. The repository can be found at:\\
\url{https://github.com/fmr42/ocl}



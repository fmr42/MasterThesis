\chapter{Further work}
\label{ch:further}
\section{UBX limitations}
The UBX framework allows to write blocks with well defined separation of concerns, but does not actually \emph{require} it: a coder could, for example, make a c-block store data in its private memory space, or make a i-block process the data with a computational behavior.\\
Moreover another limitation of UBX is that it still does not support blocks composition, while it should be possible to define sub-nodes to be used as regular blocks.\\
A new library named AB5C is in a very early developing stage and should replace UBX and solve those problematics.

\section{Missing models}
There still is a lack of modeling in this work, also because of the lack of tools to do so. More techniques are under development, especially for data; the goal is to be have (someday) developing frameworks ables to auto-generate the code from models; although there already are tool-chains able to declare (not define!) functions from models, the road to tools able to implement components from mathematical models is still long. By now, the more feasible goal is to have a centralized repository of data models (e.g. positions, rotations, ecc.) to conforms to and at least have data-model transformation tool able to generate "adapter" components to trivially connect components not developed for being connected.\\

\section{Open-source contributions}
As a personal opinion, a large users base is essential to reach high goals and the open source software developing model has proved to work once reached a "critical mass" of developers; as an example, take the ROS project: an environment very similar to UBX that is leading to success thank to a wide adoption in the robotic community.\\
Moreover, universities represent a great source of work force, especially for the complex job of modeling, and should take the initiative in order to tweak a wider (industrial, scientific, recreational, commercial) adoption.


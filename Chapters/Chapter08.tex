\chapter{Implementation}
\label{ch:implementation}
This chapter gives an introspection of the UBX project and the derived work. There are 3 section: one fore the core libraries, necessary to run nodes, one for the blocks from the official project and one for the blocks designed and implemented in the context of this thesis.

\section{Core UBX library}
This library is composed by the core shared library \texttt{libubx.so} and the Lua scripts and macros necessary to link the component functions together. Please note that since all the UBX functionalities are implemented in C or C++ function, the native Lua interpreter is not capable of linking them; it fact, the LuaJIT interpreter has been used.
\graffito{On \autoref{ch:deploy} Orocos has been used to deploy the node since it supports Lua components, the Orocos Lua interpreter however if currently the native one, for this reason the project itself has been forked and re-linked to the LuaJIT library to allow execution of a MicroBLX node on top of Orocos.}
LuaJIT implements the FFI library which allows calling external C functions and using C data structures from pure Lua code\autocite{bib:luaffi}.\\
The \texttt{Makefile} has been expanded beyond the compilation directives already present in the official UBX project: it also contains directives to install the core libraries, the scripts and the macro into the system, plus generates a proper configuration file automatically loaded by the \texttt{ubx\_launch} command.

\section{Official UBX Team blocks}
The \texttt{ubx-base} projects contains blocks from the official UBX team; some of those blocks are not used or are re-implemented while others are fundamental for the simulation to work. Moreover it contains the native (basic) data types like \texttt{int}, \texttt{char}, \texttt{long}, \texttt{long long}, \texttt{short}, \texttt{double} and \texttt{float}.
It worths to highlight the presence of two blocks:
\begin{itemize}
	\item \texttt{ptrig}: an s-block that, once properly configured, trigs a list of c-blocks in a timed loop;
	\item \texttt{webif}: this c-block does not need to be triggered nor to be connected: it is able to open a html interface useful to configure, control and monitor the node it is added to.
\end{itemize}

\section{Study-case blocks}
This section describes the blocks designed for the purpose of controlling a quadcopter.

\subsection{Copter controller (\texttt{copter/ctrl\_copter})}
This block implements the control algorithm as explained in \autoref{ch:controlalgorithm} and \autocite{marconi}. It reads the state and the set-point of the vehicle and writes a control action in the form of vectored-thrust control action (i.e. the scalar thrust and the vectored torque the propellers should generate).\\
It is implemented in C++ in order to exploit the capability of the mathematical library Armadillo that greatly simplify operations on vectors and matrices. 

\subsection{Extended predict-only Kalman filter (\texttt{copter/kf\_predict\_copter})}
Models the behavior of the vehicle: reads the estimate of the state and the control action and generate a new estimate of the state (which is usually written back to the same i-block the state was read from).\\
This block is written in C++ and use Armadillo too.

\subsection{Update-only Kalman filter (\texttt{control/kf\_update})}
Fuses the sensor data with the state. Using the notation from Wikipedia\autocite{bib:wiki:kalmann}, the sensor model matrix $ H $ is read from a file specified in the UBX node configuration file while the sensor data covariance matrix $ R $ is computed on-line from the covariance embedded in the sensor data.

\subsection{Fast shm memory access (\texttt{control/shm})}
This general-purpose i-block is able to dynamically allocate vectors of arbitrary length of arbitrary data-types in shared memory. Is is used for fast data passing between c-blocks. Data can be accessed by blocks in the same node by properly linking the port connection.\\
Note that, if necessary, multiple shm blocks running on different nodes on the same machine can share the same memory pool, however no synchronization mechanism has been implementer to avoid race condition and therefore it has been chosen to grant write permissions only to the block that allocated the pool while other blocks do only have read access.

\subsection{AHRS+ sensor data acquisition (\texttt{control/sensor\_myAHRS})}
This i-block acts as bridge between the node and the physical Hardkernel AHRS+ sensor. It is worth to point out that the official library from the sensor manufacturer can not be used with the \texttt{extern "C"} directive that is indeed necessary to run C++ blocks in UBX, so a wrapper has been created.

\subsection{GUI (\texttt{copter/copter\_graphic\_dump})}
Even if there is a dump of the state of the controller to the terminal, a graphical output is present too. This is a c-block that when triggered reads the state and the set-point of the vehicle to generate and display a 2-D output. The block use the OpenCV library for the rendering. The block obviously refuse to start out of a graphic environment.\\
The output produced is a view of the vehicle from the top, the size and the center of the environment in wich the copter is moving can be adjusted, the current copter state and set-point is rendered respectively as a red and a green circle with a small tooth to visualize the heading. More detailed information are showed in text format on the top of the screen.

\subsection{Set-point generation (\texttt{copter/prim\_trj\_gen})}
The set point of the vehicle is generated by this block from 10 primitives trajectories. The trajectory code is read from a TCP socked, then the primitive trajectory is scaled on the basis of configurable parameters both in time and space and generated as a function of time. The trajectory primitive are, by now, to translate along the 4 cardinal directions, move up and down and rotate by 90 degrees clockwise or anti-clockwise.\\
The code of the primitive is passed via a TCP connection by a GUI application described in \autoref{sec:fakeplanner}, the connection is handled using the ZMQ\cite{bib:ZMQ} library.

\section{Virtual remote controller}
\label{sec:fakeplanner}
 This GUI application act as remote controller for the copter, however it does not directly generates the set-point for the vehicle, which by the way may be unfeasible because of timing constraints, but instead instructs the trajectory generation block about which pre-defined trajectory it should generate. This virtual remote controller is named "Fake Planner" because it can (should!) be replaced by a software planner (or supervisor) capable of decision taking.
 
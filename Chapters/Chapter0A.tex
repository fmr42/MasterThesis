% Appendix A

\chapter{Quaternions}
\label{ch:app:quaternions} 
Quaternions are a number systems that extends complex numbers introduced by William Rowan Hamilton in 1843. They are commonly represented in the form
\[ q = w + x\mathbf{i} + y\mathbf{j} + z\mathbf{k} \]
where $ \mathbf{i},\ \mathbf{j}\ and\ \mathbf{k} $ are the fundamental quaternion units.
Multiplications between quaternions are non-commutative, with
\begin{table}[]
	\centering
	\caption{My caption}
	\label{my-label}
	\begin{tabular}{|l|l|}
		\hline
		\[ \mathbf{j} x \mathbf{k} = \mathbf{i} \] & \[ \mathbf{j} x \mathbf{k} = \mathbf{i} \] \\ \hline
	\end{tabular}
\end{table}

\[ \mathbf{i} x \mathbf{1} = \mathbf{i} \]
\[ \mathbf{i} x \mathbf{i} = \mathbf{-1} \]
\[ \mathbf{i} x \mathbf{j} = \mathbf{k} \]
\[ \mathbf{i} x \mathbf{k} = \mathbf{-j} \]



\[ \mathbf{k} x \mathbf{1} = \mathbf{i} \]
\[ \mathbf{k} x \mathbf{i} = \mathbf{i} \]
\[ \mathbf{k} x \mathbf{j} = \mathbf{i} \]
\[ \mathbf{k} x \mathbf{k} = \mathbf{i} \]


\section{Rotations}
Normal quaternions are

% Abstract

%\renewcommand{\abstractname}{Abstract} % Uncomment to change the name of the abstract

\pdfbookmark[1]{Abstract}{Abstract} % Bookmark name visible in a PDF viewer

\begingroup
\let\clearpage\relax
\let\cleardoublepage\relax
\let\cleardoublepage\relax

\chapter*{Abstract}
Modern engineering faces the problem of implementing systems in a flexible, scalable and reusable way. This problem becomes exponentially hard as the complexity of systems increase, unless systems and systems components developers foresees and find ways to overcome such scaling problematic.

Many techniques try to address the scaling problem by defining paradigms to standardize or, at least, harmonize the efforts of different developers teams. Nevertheless developers tends to underestimate their importance, either because of lack of education, either because they consider overcomplicated or pointless their adoption in small to medium projects. This is indeed true sometimes: the lack of mature, open, free, supported and widely accepted tools makes the benefit one can archive from the introduction of such techniques void w.r.t. the effort required to learn and adopt them. This is a vicious cycle, however projects like ROS
\graffito{ROS is a open-source component based software framework, where components are usually written and released by third party developers for being reused, avoiding the necessity of "re-inventing the wheel"}
show how also small projects can benefit from a proper and formal approach to the problem, especially once a "critical mass" of adopters is reached.

This thesis provides a study case: the Model Driven Engineering (MDE) methodology and the 5C "separation of software behavior" paradigm are applied in the development of a novel low-level copter controller. The following chapters will briefly introduce such techniques and will show how to apply them in a real-world problem.
\\
\\
As already highlighted, such small and stand-alone application does not gain enough benefit from this approach to justify the effort required, however the purpose of this thesis is to support and popularize a certain attitude to problems approach, to release working example code, contribute to the maturity of the underlying software framework and, last but not least, to enrich the knowledge and the skills of author readers.

\endgroup			

\vfill
% Abstract

\pdfbookmark[1]{Abstract - Versione in italiano}{Abstract - Versione in italiano} % Bookmark name visible in a PDF viewer

\begingroup
\let\clearpage\relax
\let\cleardoublepage\relax
\let\cleardoublepage\relax

\chapter*{Abstract - Versione in italiano}
La moderna ingegneria affronta il problema di dover implemetare sistemi flessibili, scalabili e facilmente riutilizabili. Questo problema diventa esponenzialmente difficile al crescere della complessita dei sistemi a meno che gli sviluppatori dei sistemi e dei loro componenti non adottino una mentalità più lungimirante e non trovino modi per superare questa problematica.

Molte tecniche cercano di approcciare il problema definendo paradigmi per standardizzare o, almeno, armonizzare il lavoro di diversi gruppi di sviluppo. Ciononostante gli sviluppatori tendono a sottovalutare la loro importanza, a volte per mancanza di informazione, altre volte perchè considerano il loro impiego troppo complicato o addirittura senza senso in progetti medio-piccoli. Questo è peraltro vero a volte: la mancanza di strumenti maturi, aperti, liberi, ben supportati e diffusi rende i benefici ottenibili dall'impiego di molte di queste tecniche nulli rispetto allo sforzo necessario per la loro comprensione e introduzione. Questo è un circolo vizioso, ma progetti come ROS
\graffito{ROS è un framework software basato su componenti: le componenti sono ususalmente scritte e rilasciate da terzi per poter essere facilmente riutilizzate, evitando la necessità di dover "reinventare la ruota".}
mostrano come anche piccoli progetti possono beneficiare di un corretto approccio al problema, specialmente se si riesce a raggiungere una "massa critica" di utenti.

Questa tesi presenta un caso di sviluppo: l'ingegneria guidata dal modello e soprattutto l'adozione del paradigma di "separazione dei comportamenti" 5C sono stati applicati nello sviluppo di un nuovo controllore di basso livello per un'ampia gamma di veicoli ad ala mobile. I capitoli successivi introdurranno brevemente il lettore a queste tecniche di sviluppo e mostreranno come è possibile applicarle ad un problema reale.
\\
\\
Come già sottolineato, un`applicazione così piccola e isolata non beneficia di questo approccio tanto da giustificarne la complessità, tuttavia lo scopo è supportare e divulgare quella che l`autore considera \emph{best-practice} nell'approcciare i problemi, rilasciare codice di esempio funzionante, contribuire alla maturità del framework impiegato e, non per ultimo, arricchire la conoscenza e le capacità dell`autore e dei lettori.

\endgroup			

\vfill
